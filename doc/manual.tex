\documentclass{scrartcl}

\usepackage{url}
\usepackage{hyperref}
\usepackage{amsmath,amssymb}
\usepackage[parfill]{parskip}


\title{Manual to AnA-FiTS}
\author{Andre J. Aberer}
\date{\today}


\newcommand{\console}[1]{\newline {\footnotesize \texttt{\$ #1}} \newline}
\newcommand{\prog}[1]{\texttt{#1}}




\begin{document}

\maketitle



\begin{abstract}
  This is the manual to AnA-FiTS, the ancestry-aware forward-in-time
  simulator.  AnA-FiTS simulates genetic sequences of a population
  forward in time under a Fisher-Wright model with selection and
  dynamically changing population size. Currently, the feature set is
  not particularly extensive: the development model is to implement
  advanced features on demand. Instead, AnA-FiTS primarily focuses on
  runtime speed and to the best of my knowledge outperforms competing
  software by one to two order(s) of magnitude. In addition to a
  \prog{ms}-like sequence output, AnA-FiTS provides the entire
  observable history of the sequences' neutral part (see
  Sect.~\ref{sec:quick-start}), a structure similar to the ancestrial
  recombination graph. 
\end{abstract}

\newpage 

\tableofcontents

\newpage 

\section{Installation}
\label{sec:installation}

For the installation of AnA-FiTS, please use a fairly recent \prog{g++}
compiler ($\geq$ 4.6). Other compilers may work, but have not been
tested. 

Another requirement is the \textbf{boost-library}. In specific, only
the \prog{ProgramOptions} package is required. On most ubuntu/debian
system, installing boost is as simple as 
\console{   sudo apt-get install  libboost-all-dev }
Otherwise just visit \url{http://www.boost.org/}. 

Finally, an installation of python may be required (however this is
not mandatory). 

For compiling \prog{AnA-FiTS}, enter \console{ make} on the command
line.  An executable name \prog{AnA-FiTS-\textbf{git-version}} is
created. The \textit{git-version} suffix in the name of the executable
helps you (and me) to exactly identify which version of the code was
used (in case of bugs or for reproducibility).

An executable \prog{convertSeq} and \prog{convertGraph} should be
produced by your \prog{make} call. This tiny tool converts the binary
output of AnA-FiTS into human-readable output.


\section{Quick Start}
\label{sec:quick-start}

To see AnA-Fits in action, just enter 
\console{ ./AnA-FiTS-v0.9-1-gfc23a8c  -n myRun -s 3 }
on the command line. With two mandatory command line
arguments you provide a run id (\prog{-n}) and a seed for random
number generation (\prog{-s}). All further simulation parameters are
set to default values: in the case above, we simulate a chromosome
(with length 10 Mbp) for an effective population size ($N_{e}$) of 500
diploid individuals and a per-site per-sequence recombination/mutation
rate of 1e-8. 10 \% of the mutations simulated will be under selection
(with parameter -W \prog{=2 0.05 10 2000 0.05 1 500}, see
Sect.~\ref{sec:walk-through-options} for details).

\subsection{Output}
\label{sec:output}

AnA-FiTS produces three output files. The info file contains all
information relevant to the run. In particular, this makes it possible
for you to re-create the run later. 

The file anafits\_polymorphisms.myRun contains a \prog{ms}-like
sequence representation in binary format. This means that before
having a look at the sequences data, you have to convert this file
first calling: \console{ ./convertSeq anafits\_polymorphisms.myRun}
or \console{ ./convertSeq myOutput anafits\_polymorphisms.myRun}

For the first call the output is written to your console, in the
second case, you create a file with name myOutput. 

The rationale behind the binary format is runtime speed. Formatting
the output can take up a considerable part of the total execution
time. Note, that this is a feature: if you want to run a huge number
of simulations, you can omit \prog{convertSeq} and directly parse
the binary output in your application. For instructions on how to
parse the binary format, please write  me an e-mail or try to extract
the information from src/sequenceConversion.cpp. 

Finally, AnA-FiTS also creates a graph file containing the observable
history of the neutral part of the sequence. This is a by-product of
the algorithm implemented to speed up simulation of neutral mutations,
however it also is interesting information on its own. The graph file
is in binary as well and can be converted with \console{./convertGraph ./anafits\_graph.myRun}


\subsection{Format}

The format for the polymorphism file is as follows: 
\begin{itemize}
\item line 1: chromosome id 
\item line 2: mutations that are fixed in the population (neutral and non-neutral, unordered).
  Mutations are separated by semi-colon. For each mutation, AnA-FiTS
  prints the 
  \begin{itemize}
  \item location in the sequence
  \item generation of origin (starting  with 0)
  \item selection coefficient 
  \item base (if multiple mutations at the same position occurred,
    each mutation is represented as a separate column in the matrix)
  \end{itemize}
\item line 3: contains polymorphic mutations in the same format as in
  line 2. Each mutation corresponds to one column in the matrix
  below. 
\item line 4 to beginning of next chromosome or end: contains a matrix
  representing the sequences. Each column is a polymorphic mutation,
  each row stands for a sequence. Adjacent rows (e.g., 0 and 1, 2 and
  3) are the two sequences for a diploid individual. 
\end{itemize}

In rare instances, two distinct mutation events may lead to the same
polymorphism or a back-and-forth mutation (e.g., $A \rightarrow C
\rightarrow A$) leads back to the original state. These instances are
adequately handled in AnA-FiTS, so we implement a true finite-sites
model. Since we do not want to loose information in the output, both
cases listed above yield additional columns in the polymorphism
output. In case 1, we have two adjacent columns with the same mutation
(i.e., base, location is equal), where some haplotypes have a 1 in one
column and some in the other (none of them can have both). In the
second case, haplotypes with a back-and-forth mutation will appear to
have both the mutation ($\rightarrow$ C) as well as the back-mutation
($\rightarrow$ A). So keep in mind that for these rare instances the
output does not directly represent the genotype (instead your script
has to compensate for that, but on the plus side may use this
additional information).


The format for the graph file is (see Sect~\ref{sec:graph} for further
information on the graph): 
\begin{itemize}
\item line 1: graph id (each chromosome produces a distinct graph,
  thus for organisms with two chromosomes, you will obtain two graphs)
\item line 2: ids of nodes that survived into the present generation
\item line 3 to end (resp. start of next distinct chromosome): node
  and edge information of the graph. Each entry is composed of: 
  \begin{itemize}
  \item id: the node id
  \item node description (in brackets): for neutral mutations, this is
    the location in the sequence, the generation of origin and the
    base, for recombinations this identifies the break point and the
    generation of origin
  \item edge information: id(s) of node(s) \textit{parent} haplotypes
    (two parents for a recombination, one for mutations)
  \end{itemize}   
\end{itemize}

\section{The Graph}
\label{sec:graph}

For information on how the graph can be used / interpreted, please
refer to the supplement of the main paper. Auxiliary information will
follow in this section in a future version.


\section{Important Information about Memory  / Runtime }
\label{sec:some-notes-runtime}

\subsection{Restricting Memory for Ancestry  }
\label{sec:restr-memory-ancestr}
For the most part, fast execution times of AnA-FiTS come at the cost
of highly increased memory consumption. You may find AnA-FiTS
exceeding your desktops main memory, when simulating more than 10,000
individuals. You can significantly relax memory requirements with the
\prog{-M} switch, for instance with \console{
  ./AnA-FiTS-v0.9-1-gfc23a8c -n myRun -s 3 -M 2G }

you advise AnA-FiTS not to use more than 2 GB of main memory. This,
however, is a soft constraint: for some calls, AnA-Fits will allocate
more memory nonetheless (thus, some experimentation with the parameter
may be necessary). If less memory is used, the total execution time
will definitely increase, however in many cases the performance
penalty is not severe.

Please note, if the \prog{-M} is used, the same random number seed
will not result in the exact same run for different values for
\prog{-M}. 

\subsection{Restricting Memory for Sequence Creation}
\label{sec:restr-memory-sequ}

Another tuning parameter that can be used in order to save memory is
\prog{-R} \textit{num}. The parameter influences, how much memory is
used after the BEG graph has been created (see supplementary). In
specific, it sets the number of references needed for explicitly
representing nodes in the graph. If no nodes are represented
explicitly at all (e.g. with \prog{-R} 1000), runtimes for AnA-FiTS will
significantly go up. On the other side, the more nodes are
represented, the more memory is needed. Thus, this step can become
prohibitive memory-wise. 

By default, AnA-FiTS sets this parameter automatically, such that not
more than 5 \% of the nodes of a graph are represented explicitly. If
you happen to have a particularly large graph (because of high numbers
of individuals and/or high rates), you may want to adjust this
parameter (i.e., increase it to for instance \prog{-R} 10), such that
AnA-FiTS consumes less memory at this stage (while the runtime often
is not severely increased). Note that this option does not influence
the simulation and you should always obtain the exact same result, if
you use different values of -R, but keep all further parameters
constant (if this is not the case, please report this as a bug). 



\section{Information on Program Options}
\label{sec:walk-through-options}

For most parameters, there is a short and long version, I will switch
between both possibilities. 

\subsection{General}
\label{sec:general}
\textbf{Input parameters} for AnA-Fits are not scaled.  For computing for
instance how many recombinations we expected for a given chromosome
per generation ($E[REC]$), the input value for recRate $r$ is
computed as follows: 

\begin{equation*}
  \label{eq:1}
  E[REC] = r * \textnormal{sequenceLength} * 2 * N_e. 
\end{equation*}


\subsection{Length of Simulation}
\label{sec:simulation-length}
Usually, in forward simulation, we simulate for $5 \cdot 2 \cdot N_e$
\textbf{number of generation}s (default value for AnA-FiTS). You can
influence simulation length with \prog{--SIM}.

\subsection{Population Size}
\label{sec:population-size}
Initial population size (number of diploid individuals, $N_e$) is
provided with \prog{-N}. Note that in all forward simulation, this
parameter is particularly expensive (since it also increases the total
number of generation you have simulate). You can change the
population size during simulation with the \prog{--popEvent}
options. Multiple popEvents are possible, just provide the parameter
with options multiple times. The general format is \prog{--popEvent
  <mode> <time> <args>...}, where 
\begin{itemize}
\item \textit{time} is the absolute generation number (e.g. generation
  200)
\item \textit{mode} is one of the following: 
  \begin{itemize}
  \item $c$, with argument $r$. A discrete size change with rate
    $\frac{\textnormal{new}}{\textnormal{old}}$. If you want to model
    a spontaneous population reduction by factor 2 in generation 100,
    provide \prog{--popEvent c 100 0.5}.
  \item $d$, exponential decay with rate $r$. Note that, AnA-FiTS
    rounds to the next even number of individuals.
  \item $g$, exponential growth with rate $r$ (analogously to $d$)
  \item $k$, logistic growth with rate $r$ until the carrying capacity
    $k$ is reached 
  \end{itemize}
\end{itemize}

Another example, if you want to undergo 500 individuals a bottleneck
in generation 500 that wipes out half of the generation and then
continue with logistic growth with rate 0.001 in generation 600 until
2,000 individuals are reached (however simulation ends before this is
the case), enter: \console{ ./AnA-FiTS-v0.9-1-gfc23a8c -T c 500 0.5 -T
  k 600 2000 0.001 -n run2 -s 3}


\subsection{Multiple Chromosomes }
\label{sec:mult-chrom-}

You can simulate \textbf{multiple chromosomes} (resp. unlinked loci)
with \prog{-L}. Just provide for each the chromosome the length
\console{ ./AnA-FiTS-v0.9-1-gfc23a8c -L 1000000 1000000 1000000
  1000000 1000000 -n run3 -s 3}

In the above call, 5 chromosomes, each of length 1 Mbp are
simulated. For each chromosome a separate section in the graph and
polymorphism file will be created. Elements in both files are in the
corresponding order: sequence number 1 in locus 1 occurs in the same
individual of the present generation as sequence 1 in locus 2; the
first surviving node in graph 1 belongs to the same individual as the
first surviving node in graph 2.

\subsection{Selection Model}
\label{sec:selection-model}

For simulating a neutral population, use \prog{-w}. Otherwise, you
can allow for non-neutral mutations with \prog{-W}. For runtime, the
most important thing is the fraction of mutations that are under
selection. The following modes are available: 

\begin{itemize}
\item 1 \textit{coef} $p_{pos}$ $p_{neg}$ \\
  each mutation is assign a fixed selection coefficient. The sign of
  \textit{coef} is positive with probability $p_{pos}$ (i.e. the
  mutation is deleterious) or negative with probability $p_{neg}$
  (beneficial mutation).
\item 2 $p_{pos}$ $\alpha_{1}$ $\beta_1$ $p_{neg}$ $\alpha_2$ \\
  $\beta_2$ mixture of two $\Gamma$ distributions: with probability
  $p_{pos}$ a (either deleterious or benefiicial) selection
  coefficient is drawn from $\Gamma(\alpha_1,\beta_1 )$.
\item 3  $p_{sel}$ $\mu$ $\sigma$ \\
  a mutation is non-neutral with probability $p_{sel}$ and the
  coefficient is drawn from a normal distribution with a
  \textit{positive} mean $\mu$ and standard deviation $\sigma$.
\item 4  $p_{sel}$ $\mu$ $\sigma$ \\
  a normal distribution as in 3, however mean $mu$ is negative this
  time. This means, instead of \texttt{-W 3 0.1 -0.001 0.001} you have
  to specify \texttt{-W 4 0.1 0.001 0.001}. This sub-optimal
  work-around has to be used because of a deficiency in the boost
  program options parser. 
\end{itemize}

So, if for instance we call \console{
  ./AnA-FiTS-v0.9-1-gfc23a8c -W 2 0.05 10 2000 0.05 1 500 -n run4 -s
  3}

then 10 \% of all mutations are under selection. So effectively, the
number of mutations that we have to simulate forward in time is
reduced by a factor of 10. The remaining 90\% of mutations will be
simulated using the graph based algorithm as described in the main
paper. 


\section{Validation}
\label{sec:validation}

For various cases, we compared summary statistics of data simulated
under AnA-FiTS to summary statistics simulated under either ms
(neutrality) or SFS\_CODE. You can reproduce these runs employing the
\texttt{./utils/validate.py} script. If you encounter a deviation of
sequences simulated under AnA-FiTS from the expectation, please report
this as a bug.


\section{Frequently Asked Questions}
\label{sec:freq-asked-quest}

\paragraph{AnA-FiTS outputs the entire population, why not sample? }
\label{sec:why-does-ana}

This is a feature that will be implemented at some point in the future
(will yield further runtime improvements). For now, please sample on
your own or use a quick python script like this: \console{
 ./convertSeq \textit{ana-poly} | ./utils/samplePopulation.py
 \textit{number} \textit{sampleIndi}}
Note that this script merely samples haplotypes (not individuals),
however it is consistent when used with multi-chromosome datasets
(call w/o arguments for help).

\paragraph{What should I do, if I found a bug?}
\label{sec:i-found-bug}

Please report bugs either via the github ticket system or to andre
\textit{dot} aberer \textit{at} h-its \textit{dot} org.



\section{ Acknowledgments }
\label{sec:acknowledgements}

I want to give credit to two great implementations, on which AnA-FiTS
builds: 

\begin{itemize}
\item libasmlib by Agner Fog (\url{http://www.agner.org}): I found it amazing
  that there is still vectorized integer division provided by
  Intel/AMD. Thanks to Agner for that.
\item RandomLib by Charles Karney
  (\url{http://randomlib.sourceforge.net/}): A well-crafted and
  extremely fast random number generator. I found the source code to
  be very interesting to read. 
\end{itemize}

AnA-FiTS comes with both libraries included for user convenience. If
you find the libraries to be outdated, please update from the
respective site (resp. please inform me about that).

\section{Citation and License }
\label{sec:citation}

If AnA-FiTS was useful for your scientific work, please cite as 

\begin{itemize}
\item Andre J. Aberer, Alexandros Stamatakis. AnA-FiTS: Rapid Forward
  Genome Simulation With Ancestries. 2012. \textit{unpublished.}
\end{itemize}


\noindent Note that, AnA-FiTS is released under the GNU general public
license version 3. Refer to LICENSE.txt for further information. 

\end{document}
